\documentclass[a4paper,11pt]{report}
\usepackage[T1]{fontenc}
\usepackage[utf8]{inputenc}
\usepackage{lmodern}

\title{Firedrake-Fluids User Manual}
\author{Imperial College London}

\begin{document}

\maketitle
\tableofcontents

\chapter{Introduction}
Firedrake-Fluids is a collection of numerical models for the study of fluid dynamics. It uses the Firedrake framework 

\chapter{Shallow water}
The shallow water model 

\section{Model equations}
It solves the non-linear shallow water equations, with the momentum equation in non-conservative form. The free surface is split up into a mean component $H$, and a perturbation component $h$.

\subsection{Momentum equation}
\begin{equation}
   \frac{\partial h}{\partial t} + h\mathbf{u}\cdot\nabla\mathbf{u} = -g\nabla h + \nabla\cdot\tau
\end{equation}
where $g$ is the acceleration due to gravity and $\mathbf{u}$ is the velocity. The stress tensor $\tau$ is given by 
\begin{equation}
   \nu\nabla\mathbf{u},
\end{equation}
where $\nu$ is the kinematic viscosity.

\subsection{Continuity equation}
\begin{equation}
   \frac{\partial h}{\partial t} + H\nabla\cdot\mathbf{u} = 0
\end{equation}

\section{Configuration}
The model requires three fields to be set up:
\begin{itemize}
   \item Velocity
   \item FreeSurfacePerturbation
   \item FreeSurfaceMean
\end{itemize}

\chapter{Diagnostic fields}

\section{Courant number}


\end{document}
