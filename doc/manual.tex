% Copyright 2014 Imperial College London. All rights reserved.

\documentclass[a4paper,11pt]{report}
\usepackage[top=3cm, bottom=3cm, left=3cm, right=3cm]{geometry}
\usepackage[T1]{fontenc}
\usepackage[utf8]{inputenc}
\usepackage{lmodern}

\title{Firedrake-Fluids User Manual}
\author{Imperial College London}

\begin{document}

\maketitle
\tableofcontents

\setlength{\parskip}{0.3cm}
\setlength{\parindent}{0cm}

\chapter{Introduction}
Firedrake-Fluids is a collection of numerical models for the study of fluid dynamics. It uses the Firedrake framework (\texttt{http://firedrakeproject.org/}) to automate the solution of the governing equations written in their weak form using the Unified Form Language.

\chapter{Shallow water}
The shallow water model solves the non-linear version of the shallow water equations. The free surface is split up into a mean component $H$, and a perturbation component $h$.

\section{Model equations}
The shallow water equation set comprises a momentum equation and a continuity equation, each of which are defined below.

\subsection{Momentum equation}
The momentum equation is solved in non-conservative form such that
\begin{equation}
   \frac{\partial h}{\partial t} + h\mathbf{u}\cdot\nabla\mathbf{u} = -g\nabla h + \nabla\cdot\tau - \frac{C_D||\mathbf{u}||\mathbf{u}}{H},
\end{equation}
where $g$ is the acceleration due to gravity, $\mathbf{u}$ is the velocity, and $C_D$ is the non-dimensional drag coefficient. The stress tensor $\tau$ is given by 
\begin{equation}
   \nu\nabla\mathbf{u},
\end{equation}
where $\nu$ is the kinematic viscosity.

\subsection{Continuity equation}
The continuity equation is given by
\begin{equation}
   \frac{\partial h}{\partial t} + H\nabla\cdot\mathbf{u} = 0,
\end{equation}
which assumes that $h \ll H$.

\section{Configuration}
The model requires three fields to be set up:
\begin{itemize}
   \item Velocity (a prognostic field)
   \item FreeSurfacePerturbation (a prognostic field)
   \item FreeSurfaceMean (a prescribed field)
\end{itemize}

\subsection{Drag}
To include the quadratic drag term in the momentum equation, the drag coefficient field must be enabled and the drag coefficient $C_D$ must be specified.

\subsection{Boundary conditions}
Strong Dirichlet boundary conditions can be enforced for both the FreeSurfacePerturbation and Velocity fields by selecting the \texttt{dirichlet} type. Imposing a no-normal flow condition for velocity can currently only be done weakly by integrating the continuity equation by parts and selecting the \texttt{no\_normal\_flow} boundary condition type in the configuration options.

\section{Current limitations}
\begin{itemize}
   \item Only a continuous Galerkin discretisation may be used (for all fields).
\end{itemize}

\chapter{Diagnostic fields}

\section{Courant number}
The Courant number diagnostic computes the field defined by
\begin{equation}
   \frac{||\mathbf{u}||\Delta t}{\Delta x},
\end{equation}
where $\Delta t$ is the time-step size and $\Delta x$ is the element size (more specifically, it is twice the element's circumradius).

\section{Divergence}
This diagnostic field computes the divergence
\begin{equation}
   \nabla\cdot\mathbf{u},
\end{equation}
of a vector field $\mathbf{u}$.


\end{document}
